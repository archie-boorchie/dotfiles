\documentclass[letterpaper]{article}

\usepackage{xcolor}
\usepackage{listings}
\usepackage[top=2cm,bottom=1.5cm,left=1.5cm,right=1.5cm]{geometry}

\title{Tips for an Archlinux install}
\author{Lampros Trifyllis}
\date{Dated: \today}

\begin{document}
\maketitle

\abstract{%
Here are some brief notes about a basic arch linux installation. 
These notes are to be read together with the installation guide, 
and they reflect my \emph{personal} preferences.
}

\section{Partition}
I create 4 partitions: 
\begin{enumerate}
    \item  root partition (about 25G), 
    \item  boot partition (200M), 
    \item  swap partition (about twice the ram), and 
    \item  home partition (rest of memory).  
\end{enumerate}

\noindent
{\bf Caution!} Use primary partition for 4th partition, not extended. 

\noindent
After that, format and \texttt{mount} the partitions 

\begin{lstlisting}
    mkswap /dev/sda3
    swapom /dev/sda3
    mkfs.ext4 /dev/sda1 
    mkfs.ext4 /dev/sda2 
    mkfs.ext4 /dev/sda4 
    mount /dev/sda1 /mnt
    mkdir /mnt/home
    mkdir /mnt/boot
    mount /dev/sda2 /mnt/boot
    mount /dev/sda4 /mnt/home
\end{lstlisting}
Choose a couple close mirrors and pacstrap only base to save time.
After change root, install \lstinline{reflector}, generate mirrorlist by 
\begin{lstlisting}
reflector --verbose --latest 20 --protocol https --sort rate --save /etc/pacman.d/mirrorlist
\end{lstlisting}

and install base-devel, vim, git, networkmanager, grub.

Change kernel to lts (keep the linux kernel as a backup) 
and install headers: 

pacman -S linux-headers linux-lts linux-lts-headers 
grub-mkconfig -o /boot/grub/grub.cfg
reboot



\section{Network manager}
To avoid manually configurating networks after reboot, install nm:

pacman -S networkmanager

and enable it:

systemctl enable NetworkManager

After reboot internet works out of the box. 

\section{User add}
Add a new user 

useradd -m -g wheel myname

and give him root privileges by running 
visudo and uncommentin wheel group (I use NOPASSWD option).




\section{Prettify console}
Install terminus fonts:

pacman -S terminus-font

To use terminus fonts in console do:

echo "FONT=ter-114n" >> /etc/vconsole.conf


\section{Changing default shell}
Zsh is a much more powerful shell than bash. 
To list all installed shells, run:
\begin{lstlisting}
$ chsh -l
\end{lstlisting}
and to set one as default for your user do:
\begin{lstlisting}
$ chsh -s full-path-to-shell
\end{lstlisting}
where full-path-to-shell is the full path as given by chsh -l.

If you now log out and log in again, you will be greeted by the other shell. 



\end{document}
